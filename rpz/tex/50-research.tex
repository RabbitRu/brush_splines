\chapter{Исследовательский раздел}
\label{cha:research}

\section{Условия эксперимента}
Как уже было сказано в аналитической части, мы будем ориентироваться на на работу с графическим планшетом с небольшой частотой опроса, а конкретно с Wacom Bamboo Fun CTE-650, он передаёт до 133 точек в секунду. Поскольку однозначно лучший алгоритм не был выбран, были реализованы три алгоритма: сплайн методом Акимы, алгоритм отрисовки мазками и базовый. Сделаем несколько мазков с отрисовкой базовым алгоритмом, отметками на месте полученных точек и с разной скоростью чтобы проверить наличие проблемы.
\begin{figure}
		\includegraphics[width={\textwidth}]{inc/png/test1}
		\caption{Быстро нарисованная петля}
		\label{fig:test1}
\end{figure}

\begin{figure}
		\includegraphics[width={\textwidth}]{inc/png/test2}
		\caption{Быстрый штрих}
		\label{fig:test2}
\end{figure}

Как видно на рисунках 4.1 и 4.2 просто соединяя точки прямыми при быстром ведении пера мы получим ломаную, а не гладкую кривую.%Добавить воды
\section{Тестовые данные}
Важным является вопрос эталонного результата,ведь провести одну и ту же кривую, но с разными скоростями практически невозможно, ведь человек не может идеально контролировать силу нажатия и траекторию движения пера. Однако мы можем взять математическую функцию и различными способами выбрать точки, которые в ней стоит соединять. Так как скорость проведения пером по планшету может меняться возьмём два варианта выбора точек: с фиксированным шагом аргумента, симулирующую проведение кривой без ускорения и с увеличивающимся в геометрической прогрессии с знаменателем прогрессии 1.2 аргументом, для симуляции ускорения, более быстро возрастающие прогрессии дают бессмысленно плохой результат. Наиболее реалистичным вариантом, показывающим проблему, является диапазон от 25 до 100 точек.
\subsection{Эталонная версия}
Эталонной версией будем считать версию с увеличенным на порядок количеством точек(1000), отрисованных наивным алгоритмом. С таким количеством точек расстояние между несколькими соседними получается не более нескольких пикселей и потому то, что кривая на самом деле состоит из отрезков не заметно.
\subsection{Тестовые функции, эталонные версии }
\begin{figure}
	\begin{minipage}{0.45\textwidth}
		\includegraphics[width={\textwidth}]{inc/png/idfuncs/archspiral}
		\caption{$x(t)=a*a*cos(t)$\\$y(t)=a*t*sin(t)$}
		\label{fig:func3}
	\end{minipage}
	\begin{minipage}{0,45\textwidth}
		\includegraphics[width={\textwidth}]{inc/png/idfuncs/cardioide}
		\caption{$x(t)=a(1-cos(t))cos(t)$\\$y(t)=a*sin(t)(1-cos(t))$}
		\label{fig:func4}
	\end{minipage}
\end{figure}

\begin{figure}
	\begin{minipage}{0.45\textwidth}
		\includegraphics[width={\textwidth}]{inc/png/idfuncs/deltoid}
		\caption{$x(t)=a(2cos(t)+cos(2t)$\\$y(t)=a(2sin(t)-sin(2t)$}
		\label{fig:func5}
	\end{minipage}
	\begin{minipage}{0,45\textwidth}
		\includegraphics[width={\textwidth}]{inc/png/idfuncs/ranunculo}
		\caption{$x(t)=a(6cos(t)-cos(6t))$\\$y(t)=a(6sint(t)-sin(6t)$}
		\label{fig:func6}
	\end{minipage}
\end{figure}

\begin{figure}
	\begin{minipage}{0.45\textwidth}
		\includegraphics[width={\textwidth}]{inc/png/idfuncs/sin}
		\caption{$y=sin(x)$}
		\label{fig:func1}
	\end{minipage}
	\begin{minipage}{0,45\textwidth}
		\includegraphics[width={\textwidth}]{inc/png/idfuncs/sqrt}
		\caption{$y=\sqrt{x}$}
		\label{fig:func2}
	\end{minipage}
\end{figure}


\begin{figure}
\centering
\end{figure}
\begin{figure}
\centering
\includegraphics[width={0,45\textwidth}]{inc/png/idfuncs/trifolium}
\caption{$x(t)=-acos(t)cos(3t)$\\$y(t)=-asin(t)cos(3t)$}
\label{fig:func7}
\end{figure}

\begin{table}
	[ht]
	\resizebox{\textwidth}{0.25\textheight}{
	\begin{tabular}{|p{0.18\textwidth}|p{0.28\textwidth}|p{0.4\textwidth}|}
		\hline
		Эксперимент     & Функция & Расстояние между точками\\
		\hline
		1 & Спираль архимеда & Геометрическая прогрессия\\ \hline
		2 & Спираль архимеда & Фиксированное\\ \hline
		3 & Кардиоида & Геометрическая прогрессия\\ \hline
		4 & Кардиоида & Фиксированное\\ \hline
		5 & Дельтоид & Геометрическая прогрессия\\ \hline
		6 & Дельтоид & Фиксированное\\ \hline
		7 & Ранункулоид & Геометрическая прогрессия\\ \hline
		8 & Ранункулоид & Фиксированное\\ \hline
		9 & Синус & Геометрическая прогрессия\\ \hline
		10 & Синус & Фиксированное\\ \hline
		11 & Квадратный корень & Геометрическая прогрессия\\ \hline
		12 & Квадратный корень & Фиксированное\\ \hline
		13 & Трифолиум & Геометрическая прогрессия\\ \hline
		14 & Трифолиум &Фиксированное\\ 
		\hline
	\end{tabular}
	}
	\caption{Описание экспериментов}
	\label{tab:tabular02}
\end{table}

\subsection{Алгоритм сравнения}
Для сравнения по скорости выполнения выполним каждый из вариантов комбинации алгоритм/набор точек 100 раз, для усреднения результата и замерим суммарное время отрисовки в миллисекундах. Для сравнения совпадения с эталоном мы будем попиксельно сравнивать изображения, подсчитывая число пикселей закрашенных цветом, отличающимся от цвета фона, в переменной $pixels$, суммировать разницу цвета в переменной $diff$, тогда процент совпадения с эталоном может быть посчитана по формуле $\frac{diff}{pixels} * 100\%$

\section{Результаты экспериментов}
В диаграммах совпадения чётные номера экспериментов имеют линейный шаг ускорения, а нечётные шаг с ускорением. В случае шага с ускорением, точек меньше: 18 и 22 против 30 и 60 в случае с линейным шагом т.к. нам нужен одинаковый начальный шаг, а прогрессия с увеличивающимся шагом, очевидно, проходит один и тот же период быстрее. Посчитать количество точек в случае геометрической прогрессии можно по формуле $n=1+\log_{q}(\frac{b_n}{b_1})$, где $b_n$ количество точек при линейном шаге, а $b_1$ начальное значение шага. Полный список результатов экспериментов смотри в приложении 1.
\begin{figure}
	\begin{minipage}{0.45\textwidth}
		\includegraphics[width={\textwidth}]{inc/png/res/CardiodeGeomBas}
		\caption{Базовый алгоритм на 18 точках с ускорением}
		\label{fig:res1}
	\end{minipage}
	\begin{minipage}{0,45\textwidth}
		\includegraphics[width={\textwidth}]{inc/png/res/CardiodeGeomSpl}
		\caption{Сплайн на 18 точках с ускорением}
		\label{fig:res2}
	\end{minipage}
	\begin{minipage}{0.45\textwidth}
		\includegraphics[width={\textwidth}]{inc/png/res/ArchSpiralGeomBas}
		\caption{Базовый алгоритм на 18 точках с ускорением}
		\label{fig:res3}
	\end{minipage}
	\begin{minipage}{0,45\textwidth}
		\includegraphics[width={\textwidth}]{inc/png/res/ArchSpiralGeomSpl}
		\caption{Сплайн на 18 точках с ускорением}
		\label{fig:res4}
	\end{minipage}
\end{figure}
\clearpage
\begin{figure}
	\centering
	\includegraphics[width={\textwidth},height={0,4\textheight}]{inc/png/diags/30dts}
	\caption{Диаграмма совпадений для 30 точек}
	\label{fig:diag1}
	\includegraphics[width={\textwidth},height={0,4\textheight}]{inc/png/diags/60dts}
	\caption{Диаграмма совпадений для 60 точек}
	\label{fig:diag2}
\end{figure}

\begin{figure}
	\centering
	\includegraphics[width={0,9\textwidth}]{inc/png/diags/30dtsperf}
	\caption{Диаграмма производительности для 30 точек}
	\label{fig:diag3}
\end{figure}
\begin{figure}
	\centering
	\includegraphics[width={0,9\textwidth}]{inc/png/diags/60dtsperf}
	\caption{Диаграмма производительности для 60 точек}
	\label{fig:diag4}
\end{figure}
\clearpage

\section{Выводы}
Как мы видим, в значительной части случаев все три алгоритма показывают схожие результаты совпадения с эталонной версией, однако достаточно часто сплайн показывает заметно больший процент совпадения с эталонной версией(до 30\% разницы с базовым алгоритмом), но при этом по скорости он на порядок проигрывает базовому, а алгоритму мазками в зависимости от эксперимента проигрывает в производительности от нескольких процентов до 5 раз, однако даже в худшем случае мы получаем менее миллисекунды на построение связи между двумя точками, значит его производительность достаточна для комфортной интерактивной работы. 


\section{Заключение}
В рамках данной работы нами был проведён анализ существующих алгоритмов, используемых при соединении точек, три из них были успешно модифицированы и реализованы, после чего был проведено исследование скорости их работы и совпадения с эталонной версией отрисовки. Также в рамках исследования были подтверждены теоретические преимущества и недостатки алгоритмов. Разработанное ПО успешно выполняет поставленную задачу - отрисовку линий, вводимых с помощью графического планшета.\\
Основными возможными векторами дальнейшего развития данного ПО являются: 
\begin{enumerate}
	\item	Дальнейшее исследование работы со сплайнами
	\item	Добавление других динамических параметров
	\item	Поддержка графических планшетов с датчиком наклона пера
	\item	Поддержка других устройств ввода
	\item	Симуляция реалистичных кистей
\end{enumerate}



