\chapter{Конструкторский раздел}
\label{cha:impl}

\section{Алгоритм отрисовки сплайнами }
Как уже было сказано в аналитической части, оптимальным вариантом сплайна является кубический сплайн, однако у него есть несколько модификаций, для начала рассмотрим общий вид кусочно кубической интерполяции.\cite{bg98} Для сплайна нам нужно чтобы были заданы точки $a = x_1<x_2<...<x_n=b$ и соответствующие им значения $f(x_1),f(x_2)...,f(x_n)$, по ним строится интерполирующая функция $Pf$ таким образом что на каждом отрезке $[x_i,x_i+1], y=1,...,n-1$ она является многочленом $P_i$ степени 3, таким, что 
\begin{center}
	$\begin{array}{c}
	P_i(x_i)=f(x_i),\quad P_i(x_i+1)=f(x_i+1)\\
	P_i`(x_i)=d_i, \quad_i`(x_i+1)=d_i+1 \\
	\end{array}\Bigg\} i= 1,...,n-1$\\
\end{center}
где $d_i, i=1,...,n$ - свободные параметры, тот или иной способ выбора которых определяет метод кусочной интерполяции кубическими многочленами. Полученная функция $Pf$ совпадает с $f$ в точках $x_i, i=1,..,n$ и для любого набора параметров $d_i Pf \in C^{(1)}([a,b])$.\\
Коэффициенты многочлена $P_i$, записанного в форме
\begin{center}
	$P_i(x)=a_{1,i} + a_{2,i}(x-x_i)+a_{3,i}(x-x_i)^2+a_{4,i}(x-x_i)^2(x-x_{i+1})$
\end{center} 
могут быть вычислены по интерполяционной формуле Ньютона с кратными узлами:
\begin{figure}
	[ht]
	\centering
	\includegraphics[width={0,8\textwidth}]{inc/png/nuton}
\end{figure} 
\\Отсюда получаем
\begin{center}
	$\begin{array}{l}
	a_{1,i}=f(x_i)\\
	a_{2,i}=d_i\\
	a_{3,i}=\frac{f(x_i;x_{i+1})-d_i}{x_{i+1} - x_i}\\
	a_{4,i}=\frac{d_i +d_{i+1} - 2f(x_i;x_{i+1})}{(x_{i+1}-x_i)^2}\\
	\end{array}$
\end{center}
, где $f(x_i;x_j) = \frac{f(x_j)-f(x_i)}{x_j-x_i}$ - разделённая разница.\\
Перейдём к другому виду
\begin{center}
	$P_i(x)=c_{1,i} + c_{2,i}(x-x_i)+c_{3,i}(x-x_i)^2+c_{4,i}(x-x_i)^3$
\end{center} 
, где 
\begin{center}
	$\begin{array}{l}
	c_{1,i}=a_{1,i}=f(x_i)\\
	c_{2,i}=a_{2,i}=d_i\\
	c_{3,i}=a_{3,i} - a_{4,i}(x_{i+1}-_i)=\frac{3f(x_i;x_{i+1})-2d_i-d_{i+1}}{x_{i+1} - x_i}\\
	c_{4,i}=a_{4,i}=\frac{d_i +d_{i+1} - 2f(x_i;x_{i+1})}{(x_{i+1}-x_i)^2}\\
	\end{array}$
\end{center}
\subsection{Метод Акимы}
Этот метод приближения используется для борьбы с выбросами приближающей функции, которые появляются, если значения функции в точках заданы с некоторой погрешностью. Поскольку выбросы нежелательны, а разница по сложности вычислений с простым кубическим сплайном незначительна, мы будем использовать этот метод.\cite{aspline70}\cite{comp07}\\
\par Разделённая разность $f(x_{i-1,x_i})$ является приближением к $f`(x_i)$ слева, а $f(x_i,x_{i+1})$ является приближением к $f`(x_i)$ справа. В методе Акимы эти приближения усредняются с весам, котрые тем больше, чем меньше гладкость функции на соседнем отрезке. Окончательная формула для определения параметра $d_i$ имеет вид\\

\begin{center}
	$d_i=
	\begin{cases}
	\frac{w_{i+1}f(x_{i-1};x_i)+w_{i-1}f(x_i;x_{i+1})}{w_{i+1}+w_{i-1}}, \mbox{ если } w_{i+1}^2+w_{i-1}^2 \neq 0\\
	\frac{(x_{i+1}-x_i)f(x_{i-1};x_i)+(x_i-x_{i-1})f(x_i;x_{i+1})}{x_{i+1}-x_{i-1}}, \mbox{ если }  w_{i+1}=w_{i-1}=0
	\end{cases}$
\end{center}
, где $i=3,4,...,n-2$ и $w_j=|f(x_j;x_{j+1})-f(x_{j-1};x_j)|$\\
Для получения недостающих значений $d_1,d_2,d_{n-1},d_n$ точки экстраполируются по формулам.
\begin{center}
	\qquad
	$\begin{array}{l}
	x_0=2x_2-x_4\\
	y_0=(x_1-x_0)(f(x_3;x_2)-2f(x_2;x_1))+y_1\\
	x_1=x_2+x_3-x_4\\
	y_1=(x_4-x_3)(f(x_4;x_3)-2f(x_3;x_2))+y_2\\
	x_{n-2}=x_{n-3}+x_{n-4}-x_{n-5}\\
	y_{n-2}=(x_{n-2}-x_{n-3})(2f(x_{n-3};x_{n-4})-f(x_{n-4};x_{n-3}))+y_{n-3}\\
	x_{n-2}=2x_{n-3}-x_{n-5}\\
	y_{n-2}=(x_{n-1}-x_{n-2})(2f(x_{n-2};x_{n-3})-f(x_{n-3};x_{n-2}))+y_{n-2}\\
	\end{array}$
\end{center}
\subsection{Проблема использования сплайнов при рисовании}
Основное применение сплайнов в компьютерной графике - построение графиков и с этим они хорошо справляются, однако при использовании сплайнов в рисовании у нас могут не выполнятся условия упорядоченности точек по оси Ох и как следствие сплайн не может быть построен. Кроме того, в условиях сильно различающихся изменений по осям Ох и Оу сплайн может рисовать кривую с слишком большим изгибом. Следовательно требуется модифицировать алгоритм рисования сплайна.
\subsection{Предлагаемая модификация }
Модификация должна решать две проблемы:\\
\par Первая - не по каждому набору точке можно построить сплайн, необходимо чтобы по одной из координат они не совпадали и были упорядочены по возрастанию, если представить точки в таком виде возможно, то строится сплайн, иначе точки соединяются прямыми линиями.
\begin{figure}
	\centering
	\includegraphics[width={0,25\textwidth}]{inc/png/badspline}
	\caption{Пример точек, по которым невозможно построить сплайн}
	\label{fig:fig07}
\end{figure} 
\\ \par Вторая - в результате построения сплайна мы можем получить кривую, которая будет иметь сильный изгиб, так называемый выброс, такой результат отрисовывать не стоит, поэтому проверяется выходит ли кривая за границы наименьшего прямоугольника, описывающего входные три точки и рисуется только если не выходит.\\\
\begin{figure}
	\centering
	\includegraphics[width={0,4\textwidth}]{inc/png/badspline2}
	\caption{Выброс}
	\label{fig:fig08}
\end{figure} 
\begin{figure}
	\centering
	\includegraphics[width={0,8\textwidth}]{inc/png/splinediag}
	\caption{Блоксхема модификации}
	\label{fig:fig09}
\end{figure} 


\section{Базовый алгоритм}
В случае задания мазка, проведения им линии до следующей точки при использовании прозрачной кисти мы получим артефакты в виде вдвое меньшей прозрачности в точках из-за наложения линий одной на другую, поэтому алгоритм требуется модифицировать.
\begin{figure}
	\centering
	\includegraphics[width=\textwidth]{inc/png/bas1}
	\caption{Артефакты наложения линий}
	\label{fig:fig10}
\end{figure} 
\subsection{Предлагаемая модификация}
Можно затирать конец прошлой линии, перед рисованием следующей, однако, в таком случае артефакты появляются при пересечении линий.
\begin{lstlisting}[style=pseudocode,caption={Модификация базового алгоритма}]
if (Transparency not full):
	drawPoint(Previous Point, Background Colour)
drawline(Previous Point, Current Point)
\end{lstlisting}

\begin{figure}
	[ht]
	\centering
	\includegraphics[width=\textwidth]{inc/png/basmod}
	\caption{Артефакты модификации}
	\label{fig:fig11}
\end{figure} 
\section{Алгоритм отрисовки мазками}
Для отрисовки непрозрачной кистью будем использовать алгоритм с модификацией отступа 25\%, а для прозрачной кисти наивный алгоритм для баланса между красотой картинки и скоростью работы.

\begin{figure}
	
	\centering
	\includegraphics[width={0,8\textwidth}]{inc/png/moddiag}
	\caption{Алгоритм рисования мазками}
	\label{fig:fig12}
\end{figure} 



