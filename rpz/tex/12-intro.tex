\Introduction


Современные дизайнеры и художники, как профессионалы, так и любители используют для рисования графические планшеты и соответствующее программное обеспечение. Частота опроса традиционного устройства ввода - usb-мыши может колебаться от 125 Гц до 1 кГц, то есть, до тысячи точек в секунду, в случае графических планшетов диапазон аналогичен, однако планшеты с большой частотой достаточно дороги, поэтому будем считать что частота обновления не больше 125 Гц, то есть мы получаем точку не чаще каждых 8 миллисекунд \cite{ps2}. Одним из базовых инструментов в программах, используемых для рисования с помощью графического планшета является инструмент \textquotedblleftДинамическая кисть\textquotedblright , однако его реализация зависит от преследуемых целей, таких как поддерживаемые параметры, требуемая скорость работы, поддержка различных устройств ввода, реалистичность симулирования некого реального инструмента, такого как китайская кисть, акварельная краска, пастель или что-то ещё. Все графические планшеты поддерживают определение давления пером и наиболее популярными динамическими параметрами являются изменение толщины кисти и её прозрачности, в зависимости от давления. В рамках данной работы будут рассмотренны особенности работы с графическим планшетом и реализованы базовые динамические параметры - изменение толщины кисти и её прозрачности в зависимости от давления, кроме того необходимо будет исследовать скорость работы, ведь рисование на графическом планшете это интерактивная задача.
Для достижения поставленной цели необходимо решить следующие задачи:

\begin{itemize}
\item анализ алгоритмов рисования динамической кистью;
\item реализация выбранных алгоритмов;
\item экспериментальная проверка их работы;
\item анализ результатов;

\end{itemize}

%Вот так-то. А этот абзац вставлен для визуальной оценки отступа от перечня до следующего абзаца.