\chapter{Технологический раздел}
\label{cha:design}

\section{Выбор парадигмы и языка программирования }

Очевидным выбором для реализации программы стал объектно-ориентированный подход. Главным преимуществом этого подхода, в отличие от структурного, стала возможность не вносить изменений в законченный, рабочий код, а также по мере необходимости делать надстройки над готовой программой. Еще одним плюсом технологии является полиморфизм – возможность использовать любой объект в рамках эксплуатации программного продукта.
Языком, подходящим под поставленную задачу, был выбран С++. Этот язык имеет большое количество возможностей, которые позволяют разрабатывать надежные программные продукты. \\
Язык С++ обладает следующими достоинствами:
\begin{enumerate}
	\item	поддержка объектно-ориентированного подхода, что обеспечивает полиморфизм, наследование и инкапсуляцию при реализации задачи
	\item	использование шаблонов для создания обобщенных контейнеров для разных типов данных
	\item	автоматический вызов деструкторов при уничтожении объектов в порядке, обратном их созданию
	\item	совместимость с языком С, курс по которому был пройден ранее (использовании библиотек на С, поддержка кода на С)
	\item	контроль над структурами и порядком выполнения программы
	\item	наличие перегрузки операторов
\end{enumerate}
Кроме того, язык С++ является самым быстрым среди объектно-ориентированных языков и позволяет при необходимости выполнять достаточно низкоуровневые операции, а также напрямую работать с памятью.
\section{Выбор среды разработки }

В качестве среды разработки был выбран Qt Creator, так как на данный момент он является одной из новейших платформ, отвечающих требованиям проектирования и создания современного ПО, полностью совместим с языком С++, а также библиотеки Qt кроссплатформенные, что позволяет разрабатывать проект как под Microsoft Windows, так и под Linux.


\section{Особенности работы с графическим планшетом}
Когда с планшетом происходит какое-то действие - нажатие на него стилусом, нажатие кнопки на нём или что-то ещё, то он вызывает обрабочик событий и передаёт ему все необходимые данные. В библиотеке Qt обработчику передаётся объект класса QTabletEvent, содержащий всю необходимую информацию о действии. Особенно важными являются методы: pos() - сообщает позицию пера в экранных координатах, pressure - сообщающий уровень давления, type() - позволяющий определить это начало, процесс или конец движения пера по планшету и pointerType() - различающий грифель и ластик.
 
