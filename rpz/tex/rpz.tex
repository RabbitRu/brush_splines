%% Преамбула TeX-файла
% 1. Стиль и язык
\documentclass[utf8x, 14pt]{G7-32} % Стиль (по умолчанию будет 14pt)

% Остальные стандартные настройки убраны в preamble.inc.tex.
\sloppy

% Настройки стиля ГОСТ 7-32
% Для начала определяем, хотим мы или нет, чтобы рисунки и таблицы нумеровались в пределах раздела, или нам нужна сквозная нумерация.
\EqInChapter % формулы будут нумероваться в пределах раздела
\TableInChapter % таблицы будут нумероваться в пределах раздела
\PicInChapter % рисунки будут нумероваться в пределах раздела

% Добавляем гипертекстовое оглавление в PDF
\usepackage[
bookmarks=true, colorlinks=true, unicode=true,
urlcolor=black,linkcolor=black, anchorcolor=black,
citecolor=black, menucolor=black, filecolor=black,
]{hyperref}

% Изменение начертания шрифта --- после чего выглядит таймсоподобно.
% apt-get install scalable-cyrfonts-tex

\IfFileExists{cyrtimes.sty}
    {
        \usepackage{cyrtimespatched}
    }
    {
        % А если Times нету, то будет CM...
    }

\usepackage{graphicx}   % Пакет для включения рисунков

% С такими оно полями оно работает по-умолчанию:
% \RequirePackage[left=20mm,right=10mm,top=20mm,bottom=20mm,headsep=0pt]{geometry}
% Если вас тошнит от поля в 10мм --- увеличивайте до 20-ти, ну и про переплёт не забывайте:
\geometry{right=10mm}
\geometry{left=30mm}

\usepackage{pgfplots}
% Пакет Tikz
\usepackage{tikz}
\usetikzlibrary{arrows,positioning,shadows}

% Произвольная нумерация списков.
\usepackage{enumerate}

% ячейки в несколько строчек
\usepackage{multirow}

% itemize внутри tabular
\usepackage{paralist,array}


% Настройки листингов.
\ifPDFTeX
% 8 Листинги

\usepackage{listings}

% Значения по умолчанию
\lstset{
  basicstyle= \footnotesize,
  breakatwhitespace=true,% разрыв строк только на whitespacce
  breaklines=true,       % переносить длинные строки
%   captionpos=b,          % подписи снизу -- вроде не надо
  inputencoding=koi8-r,
  numbers=left,          % нумерация слева
  numberstyle=\footnotesize,
  showspaces=false,      % показывать пробелы подчеркиваниями -- идиотизм 70-х годов
  showstringspaces=false,
  showtabs=false,        % и табы тоже
  stepnumber=1,
  tabsize=4,              % кому нужны табы по 8 символов?
  frame=single
}

% Стиль для псевдокода: строчки обычно короткие, поэтому размер шрифта побольше
\lstdefinestyle{pseudocode}{
  basicstyle=\small,
  keywordstyle=\color{black}\bfseries\underbar,
  language=Pseudocode,
  numberstyle=\footnotesize,
  commentstyle=\footnotesize\it
}

% Стиль для обычного кода: маленький шрифт
\lstdefinestyle{realcode}{
  basicstyle=\scriptsize,
  numberstyle=\footnotesize
}

% Стиль для коротких кусков обычного кода: средний шрифт
\lstdefinestyle{simplecode}{
  basicstyle=\footnotesize,
  numberstyle=\footnotesize
}

% Стиль для BNF
\lstdefinestyle{grammar}{
  basicstyle=\footnotesize,
  numberstyle=\footnotesize,
  stringstyle=\bfseries\ttfamily,
  language=BNF
}

% Определим свой язык для написания псевдокодов на основе Python
\lstdefinelanguage[]{Pseudocode}[]{Python}{
  morekeywords={each,empty,wait,do},% ключевые слова добавлять сюда
  morecomment=[s]{\{}{\}},% комменты {а-ля Pascal} смотрятся нагляднее
  literate=% а сюда добавлять операторы, которые хотите отображать как мат. символы
    {->}{\ensuremath{$\rightarrow$}~}2%
    {<-}{\ensuremath{$\leftarrow$}~}2%
    {:=}{\ensuremath{$\leftarrow$}~}2%
    {<--}{\ensuremath{$\Longleftarrow$}~}2%
}[keywords,comments]

% Свой язык для задания грамматик в BNF
\lstdefinelanguage[]{BNF}[]{}{
  morekeywords={},
  morecomment=[s]{@}{@},
  morestring=[b]",%
  literate=%
    {->}{\ensuremath{$\rightarrow$}~}2%
    {*}{\ensuremath{$^*$}~}2%
    {+}{\ensuremath{$^+$}~}2%
    {|}{\ensuremath{$|$}~}2%
}[keywords,comments,strings]

% Подписи к листингам на русском языке.
\renewcommand\lstlistingname{\cyr\CYRL\cyri\cyrs\cyrt\cyri\cyrn\cyrg}
\renewcommand\lstlistlistingname{\cyr\CYRL\cyri\cyrs\cyrt\cyri\cyrn\cyrg\cyri}

\else
\usepackage{minted}

\def\mintedoptions{linenos,frame=single,style=bw}

\newminted{c}{
fontsize=\footnotesize,
\mintedoptions{}
}

\newmintedfile{python}{
fontsize=\footnotesize,
\mintedoptions{}
}

\newminted{ebnf}{
fontsize=\footnotesize,
\mintedoptions{}
}

\renewcommand{\listingscaption}{Листинг}
\renewcommand{\listoflistingscaption}{Листинги}
\fi

% Полезные макросы листингов.
% Любимые команды
\newcommand{\Code}[1]{\textbf{#1}}


\begin{document}

\frontmatter % выключает нумерацию ВСЕГО; здесь начинаются ненумерованные главы: реферат, введение, глоссарий, сокращения и прочее.

% Команды \breakingbeforechapters и \nonbreakingbeforechapters
% управляют разрывом страницы перед главами.
% По-умолчанию страница разрывается.

% \nobreakingbeforechapters
% \breakingbeforechapters

%% Также можно использовать \Referat, как в оригинале
%\begin{abstract}
%	Титульный лист. Эта страница нужна мне, чтобы не сбивалась нумерация страниц
%	\cite{Dh}
%	\cite{Bayer}
%	\cite{Habr1}
%	\cite{Noise_func}
%	\cite{Ulich}

%Это пример каркаса расчётно-пояснительной записки, желательный к использованию в РПЗ проекта по курсу РСОИ.

%Данный опус, как и более новые версии этого документа, можно взять по адресу (\url{https://github.com/rominf/latex-g7-32}).

%Текст в документе носит совершенно абстрактный характер.
%\end{abstract}
% НАЧАЛО ТИТУЛЬНОГО ЛИСТА
\begin{center}
	\hfill \break
	\textit{
		\normalsize{Государственное образовательное учреждение высшего профессионального образования}}\\ 
	
	\textit{
		\normalsize  {\bf  «Московский государственный технический университет}\\ 
		\normalsize  {\bf имени Н. Э. Баумана»}\\
		\normalsize  {\bf (МГТУ им. Н.Э. Баумана)}\\
	}
	\noindent\rule{\textwidth}{2pt}
	\hfill \break
	\noindent
	\makebox[0pt][l]{ФАКУЛЬТЕТ}%
	\makebox[\textwidth][c]{«Информатика и системы управления»}%
	\\
	\noindent
	\makebox[0pt][l]{КАФЕДРА}%
	\makebox[\textwidth][r]{«Программное обеспечение ЭВМ и информационные технологии»}%
	\\
	\hfill\break
	\hfill \break
	\hfill \break
	\hfill \break
	\normalsize{\bf Р А С Ч Ё Т Н О - П О Я С Н И Т Е Л Ь Н А Я\space\space З А П И С К А}\\
	\normalsize{\bf к курсовой работе на тему:}\\
	\hfill \break
	\large{Реализация инструмента рисования для графических редакторов <<Динамическая кисть>>}\\
	\hfill \break
	\hfill \break
	\hfill \break
	\hfill \break
	\hfill \break
	\normalsize {
		\noindent
		\makebox[0pt][l]{Студент}%
		\makebox[\textwidth][c]{}%
		\makebox[0pt][r]{{$\underset{\text{(Подипсь, дата)}}{\underline{\hspace{8cm}}}$ \space Е.О. Домнин}}
	}\\
	\hfill \break
	
	\normalsize {
		\noindent
		\makebox[0pt][l]{Руководитель курсового проекта}%
		\makebox[\textwidth][c]{}%
		\makebox[0pt][r]{{$\underset{\text{(Подпись, дата)}}{\underline{\hspace{6cm}}}$ \space И.О. Фамилия}}
	}
	\hfill \break
	\hfill \break
	\hfill \break
	\hfill \break
\end{center}

\hfill \break
\hfill \break
\begin{center} Москва 2016 \end{center}
\thispagestyle{empty} % 

% КОНЕЦ ТИТУЛЬНОГО ЛИСТА

%%% Local Variables: 
%%% mode: latex
%%% TeX-master: "rpz"
%%% End: 


\tableofcontents

%\Defines % Необходимые определения. Вряд ли понадобться
\begin{description}
\item[сложно] очень.
\end{description}

%%% Local Variables:
%%% mode: latex
%%% TeX-master: "rpz"
%%% End:

\Abbreviations %% Список обозначений и сокращений в тексте
\begin{description}
\item[GTK+]  (сокращение от GIMP ToolKit) кроссплатформенная библиотека элементов интерфейса (фреймворк), имеет простой в использовании API, наряду с Qt является одной из двух наиболее популярных на сегодняшний день библиотек для X Window System.
\item[KDE] полнофункциональная среда рабочего стола. В рамках проекта KDE разрабатывается большое количество приложений для повседневных нужд. KDE использует библиотеки Qt.
\item[Qt] кроссплатформенный инструментарий разработки ПО на языке программирования C++.
\item[GNU GPL] (Сокращение от GNU General Public License) — лицензия на свободное программное обеспечение, созданная в рамках проекта GNU в 1988 г., по которой автор передаёт программное обеспечение в общественную собственность.
\end{description}

%%% Local Variables:
%%% mode: latex
%%% TeX-master: "rpz"
%%% End:


\Introduction


Современные дизайнеры и художники, как профессионалы, так и любители используют для рисования графические планшеты и соответствующее программное обеспечение. Частота опроса традиционного устройства ввода - usb-мыши может колебаться от 125 Гц до 1 кГц, то есть, до тысячи точек в секунду, в случае графических планшетов диапазон аналогичен, однако планшеты с большой частотой достаточно дороги, поэтому будем считать что частота обновления не больше 125 Гц, то есть мы получаем точку не чаще каждых 8 миллисекунд \cite{ps2}. Одним из базовых инструментов в программах, используемых для рисования с помощью графического планшета является инструмент \textquotedblleftДинамическая кисть\textquotedblright , однако его реализация зависит от преследуемых целей, таких как поддерживаемые параметры, требуемая скорость работы, поддержка различных устройств ввода, реалистичность симулирования некого реального инструмента, такого как китайская кисть, акварельная краска, пастель или что-то ещё. Все графические планшеты поддерживают определение давления пером и наиболее популярными динамическими параметрами являются изменение толщины кисти и её прозрачности, в зависимости от давления. В рамках данной работы будут рассмотренны особенности работы с графическим планшетом и реализованы базовые динамические параметры - изменение толщины кисти и её прозрачности в зависимости от давления, кроме того необходимо будет исследовать скорость работы, ведь рисование на графическом планшете это интерактивная задача.
Для достижения поставленной цели необходимо решить следующие задачи:

\begin{itemize}
\item анализ алгоритмов рисования динамической кистью;
\item реализация выбранных алгоритмов;
\item экспериментальная проверка их работы;
\item анализ результатов;

\end{itemize}

%Вот так-то. А этот абзац вставлен для визуальной оценки отступа от перечня до следующего абзаца.

\mainmatter % это включает нумерацию глав и секций в документе ниже

\chapter{Аналитический раздел}
\label{cha:analysis}
%
% % В начале раздела  можно напомнить его цель
%
 Есть несколько общепринятых характеристик алгоритма:\cite{vc85}
\begin{enumerate}
	\item	Количество операций с пикселями
	\item	Количество предварительных вычислений
	\item	Количество поддерживаемых динамических параметров
	\item	Деградация изображения относительно наивной реализации
\end{enumerate}
В связи с тем, что последнее время сильно выросли разрешающие способности экранов и, как следствие, выросла нагрузка на аппаратную часть, необходимо будет обратить внимание на скорость работы алгоритмов.
Под деградацией изображения понимается его отличие от того, что хотел нарисовать пользователь, получившиеся из-за особенностей работы устройства ввода и рисующего алгоритма, подробнее способ сравнения будет рассмотрен в исследовательской части.
Так как у графического планшета могут быть различные дополнительные, по сравнению с мышкой, датчики, такие как датчик давления и датчик наклона пера, параметры рисуемой линии могут меняться в зависимости от изменения их значений в процессе рисования для большей реалистичности. Возможные динамические параметры кисти:
\begin{enumerate}
	\item	Цвет
	\item	Размер
	\item	Прозрачность
	\item	Форма
\end{enumerate}

\section{Анализ существующих алгоритмов}

В качестве входных данных нам поступает некоторый набор точек, по которым мы рисуем кривую. Так что у нас есть две подзадачи: обработка набора точек и его отрисовка. Будем называть мазком результат отрисовки нашей кисти в одной точке.
%\begin{itemize}
%	\item 	
%\end{itemize}

\subsection{Базовый алгоритм}
Самым простым способом отрисовки будет рассмотрение каждых двух точек в наборе как начало и конец отрезка, которые мы соединяем проводя линию мазком который должен быть в начальной точке.\\
\begin{figure}
	[ht]
	\centering
	\includegraphics[width=\textwidth]{inc/png/naive}
	\caption{Визиты пикселей в базовой реализации}
	\label{fig:fig01}
\end{figure}\\
Обозначим длину линии соединяющей точки как $d$, а радиус круга, описывающего мазок, как $r$, тогда получаем $O(dr^2)$ визитов пикселей, $O(d)$ вычислений, деградация изображения будет заметна при достаточно большом расстоянии между точками.\\


\subsection{Наивный алгоритм отрисовки мазками}
В случае наивного алгоритма каждые две точки в наборе мы рассматриваем как начало и конец линии, с помощью алгоритма, аналогичного Брезенхему, мы выбираем из линии пиксели и каждый из них считаем центром отдельного мазка, это позволяет плавно изменять параметры между мазками. По сложности алгоритм отличается от базового незначительно $O(d((r_1+r_2)/2)^2)$ визитов пикселей, где $r_1$ это радиус мазка в начальной точке, а $r_2$ радиус мазка в конечной точке, вычислений также $O(d)$.\cite{vc85}\\
\begin{figure}
	[ht]
	\centering
	\includegraphics[width={0,35\textwidth}]{inc/png/modnice}
	\caption{Динамическое изменение толщины и прозрачности между точками}
	\label{fig:fig1}
\end{figure}\\
\subsection{Алгоритм отрисовки мазками с модификацией отступов}
В наивном алгоритме мы рисовали новый мазок на каждом следующем пикселе на пути линии, попробуем уменьшить частоту отрисовки мазков в зависимости от их размера. Слева показан результат рисования кистью в виде незаполненного круга, справа - заполненного. \\
При очень большом отступе в 200 \% радиуса мазка получаем отдельные мазки, а не линию, как видно на рисунке 1.3.
\begin{figure}
	[ht]
	\centering
	\includegraphics[width=\textwidth]{inc/png/spc200}
	\caption{Отступ 200 \% радиуса мазка}
	\label{fig:fig02}
\end{figure}\\

На рисунке 1.4 видно, что в случае относительно небольшого отступа в 25 \% радиуса мазка получаем линии, однако деградация картинки при рисовании незаполненным кругом всё ещё очень велика.
\begin{figure}
	[ht]
	\centering
	\includegraphics[width=\textwidth]{inc/png/spc25}
	\caption{Отступ 25 \% радиуса мазка}
	\label{fig:fig03}
\end{figure}\\

На рисунке 1.5 видно, что при отступе в 1 \% радиуса мазка мы получаем визуально красивые линии для обеих кистей. Такой отступ сэкономит ресурсы относительно наивной реализации для кистей с радиусом более 100 пикселей.
\begin{figure}
	[!ht]
	\centering
	\includegraphics[width=\textwidth]{inc/png/spc1}
	\caption{Отступ 1 \%  радиуса мазка}
	\label{fig:fig04}
\end{figure}

В отличие от простого алгоритма мы получаем некоторый выигрыш в скорости, но при этом происходит ухудшение качества изображения. Оптимальным вариантом является использование отступа в 25\% для непрозрачной кисти\cite{bp04}.По сложности алгоритм отличается от наивной отрисовки мазками в $min(r_1,r_2)/4$ раз по визитам пикселей, что не меняет порядок сложности в $O(d((r_1+r_2)/2)^2)$ визитов пикселей, где $r_1$ это радиус мазка в начальной точке, а $r_2$ радиус мазка в конечной точке, вычислений также $O(d)$

\subsection{Алгоритм Бидермана}
Данный алгоритм требует, чтобы кисть была выпуклым многоугольником, не поддерживает прозрачность и изменение параметров между точками. Идея заключается в том, чтобы в случае симметричной кисти представить её как кисть высотой $2r$ и шириной в один пиксель - центральный столбец кисти. Тогда в начале и конце отрезка можно нарисовать полные мазки, а между ними закрасить всё этой упрощённой кистью. Для асимметричной кисти требуется взять две кисти - горизонтальный и вертикальный разрезы. Соответственно, алгоритм прохода по линии, аналогичный Брезенхему, решает, какой из мазков использовать: при движении по горизонтали - вертикальный, при движении по вертикали - горизонтальный и при движении по диагонали - оба.\cite{vc85}

\begin{figure}
	[ht]
	\centering
	\includegraphics[width=\textwidth]{inc/png/biedermann}
	\caption{Симметричный случай Бидермана}
	\label{fig:fig05}
\end{figure}

В этом алгоритме мы получаем $O(dr)$ визитов пикселей, $O(d)$ вычислений, ухудшение качества изображения и отсутствие поддержки прозрачности.

%\subsection{Заметающий алгоритм}
%Форма обязана быть выпуклым многоугольником, а также алгоритм не поддердживает плавное изменение параметров между %точками. На краях отрисовывается по мазку, после чего создаётся матрица просчитывающая след мазка \cite{vc85}

\subsection{Алгоритм с использованием сплайнов}
В статье \cite{cgim02} рассматривается два вида сплайнов: Б-сплайны и интерполяционные сплайны. Разница заключается в том, что интерполяционный обязательно проходит через центры точек, а Б-сплайн аппроксимирует и не обязан проходить через них, однако общая идея одинаковая. Так как нам важно наиболее близкое соответствие рисунка мазку, мы будем использовать одну из разновидностей интерполяционного сплайна. На некотором расстоянии друг от друга выбираются контрольные точки, которые мы назовём узлами и в которых мы будем хранить значения различных параметров. С помощью сплайнов мы строим кривую, соединяющую точки мазков в узлах. С помощью построенной кривой получаем набор промежуточных между узлами точек, которые соединяются одним из алгоритмов соединения двух точек, линейно изменяя параметры от узла к узлу. Чем большая степень используется для полиномов в сплайне, тем больше требуется вычислений и тем сильнее он колеблется между узлами, потому оптимально использовать кубический сплайн. \cite{aspline70} Основные проблема алгоритма - сложность предварительных вычислений, которые зависят только от количества точек и возможность \textquotedblleftвыброса\textquotedblright - некрасиво построенного сплайна между точками.  Таким образом получаем  $O(NX_1)$ визитов пикселей, $O(NX_2)$ вычислений, где $N$ - количество интерполированных точек, $X_1$ - количество визитов пикселей а, $X_2$ - сложность вычислений в соединяющем их алгоритме и возможную деградацию изображения из-за \textquotedblleftвыбросов\textquotedblright относительно наивного метода. \cite{cgim02}\\ 

\begin{figure}
	[ht]
	\centering
	\includegraphics[width=\textwidth]{inc/png/spline}
	\caption{Слева видно оригинальную кривую, а справа результат её отрисовки сплайном}
	\label{fig:fig06}
\end{figure}

\section{Сравнение алгоритмов}
 $d$ - расстояние между соседними точками, $r$ - диаметр мазка, а $N$ - количество точек при интерполяции.
\begin{table}
	[ht]
	\begin{tabular}{|p{0.31\textwidth}|p{0.1\textwidth}|p{0.12\textwidth}|p{0.11\textwidth}|p{0.12\textwidth}|p{0.12\textwidth}|}
		\hline
		Алгоритм     & Базовый & Наивный & Мод. отступов  & Бидерман  & Сплайны\\
		\hline
		Операций с пикселями  & $O(dr^2)$ & $O(dr^2)$   & $O(dr^2)$    & $O(dr)$       & $O(dr^2)$         \\
		Предварит. вычислений      & $O(d)$  & $O(d)$   & $O(d)$    & $O(d)$     & $O(N)$          \\
		Поддержка всех параметров & Да   & Да   & Да    & Нет     & Да          \\
		Деградация изображения & Да & Нет  & Да  & Да     & Да          \\
		\hline
	\end{tabular}
	\caption{Сравнение алгоритмов}
	\label{tab:tabular01}
\end{table}
\section{Выводы}
В большинстве современных редакторов используется алгоритм отрисовки мазками с отступами из-за его относительно простой реализации, почти незаметного ухудшения качества изображения по сравнению с наивной реализацией при использовании небольших отступов и поддержки всех возможных параметров. Однако, он плохо подходит для рисования прозрачной кистью, поэтому представляет интерес гибрид двух алгоритмов: рисующий без отступов в случае прозрачной кисти для лучшего качества картинки и рисующий с отступами при непрозрачной кисти для лучшего быстродействия.
\par Также интересным выглядит алгоритм, использующий сплайны, поскольку из-за недостаточно высокой частоты опроса у графического планшета при быстром движении просто соединяя точки линиями мы можем увидеть что кривая состоит из нескольких отрезков, что некрасиво, а сплайн сгладит этот артефакт, однако у него велика сложность предварительных вычислений, что ставит под вопрос возможность комфортной интерактивной отрисовки.
\par Так как алгоритм Бидермана не позволяет динамически изменять параметры между точками, результат его работы будет слишком некрасив, а следовательно он не интересен, несмотря на скорость его работы.
\par Нужно определить возможно ли использование такого сложного алгоритма как отрисовка сплайнами при интерактивном вводе, требуется ли оно при рисовании с устройства ввода с относительно небольшой частотой опроса и насколько заметно отличие полученного изображения от идеала, для этого потребуется реализация  алгоритмов и сравнительные эксперименты. 

\clearpage
\section{Анализ существующих программ}
Существует достаточно много различных программ с реализациями динамических кистей, таких как относительно примитивные программы для рисования (Microsoft Paint), фоторедакторы (Adobe Photoshop) или специализированные программы для художников (Krita). Рассмотрим некоторые из них.
%Нужно бы ещё что-нибудь векторное 

\subsection{Krita}
Krita — растровый графический редактор, программное обеспечение, входящее в состав KDE как часть офисного пакета Calligra Suite. Разрабатывается преимущественно для художников и фотографов, распространяется на условиях GNU GPL.\\
Krita — Krita Foundation,2016,\\
https://krita.org/en/\\
\subsubsection{Достоинства}
\begin{enumerate}
	\item Open source.
	%\item Большой набор кистей и их параметров, импорт/экспорт, создание и измене­ние кистей и их групп. Имеются такие функции кистей как размытие, смешение цветов, стирание и т. д.
	\item Поддержка графического планшета.
	\item Программа доступна для Linux, Windows и Mac OS.
\end{enumerate}

\subsubsection{Недостатки}
\begin{enumerate}
	\item Медленная работа с кистями большого размера.
\end{enumerate}

\subsection{Adobe Photoshop}
Adobe Photoshop — многофункциональный графический редактор, разрабо­танный и распространяемый фирмой Adobe Systems. В основном работает с растро­выми изображениями, однако имеет некоторые векторные инструменты. Изначально данное ПО было разработано как редактор изображений для полиграфии, но в дан­ное время оно широко используется и в веб-дизайне.\\
Adobe Photoshop CS6 — Adobe Systems, 2016,\\
http://www.adobe.com/ru/products/photoshop.html\\

\subsubsection{Достоинства}
\begin{enumerate}
	\item Поддержка графического планшета.
	\item Тесная связь с другими продуктами Adobe Systems.
	\item Поддержка Windows, Mac OS, а также мобильных ОС Android и iOS
\end{enumerate}

\subsubsection{Недостатки}
\begin{enumerate}
	\item Высокая цена лицензии.
\end{enumerate}

\subsection{Microsoft Paint}
Microsoft Paint — многофункциональный, но в то же время довольно простой в использовании растровый графический редактор компании Microsoft, входящий в состав всех операционных систем Windows, начиная с первых версий.\\
Microsoft Paint - Microsoft Corporation, 2016,\\
https://www.microsoft.com/ru-ru/windows\\

\subsubsection{Достоинства}
\begin{enumerate}
	\item Предустановлен в Windows.
	\item Прост в пользовании.
\end{enumerate}

\subsubsection{Недостатки}
\begin{enumerate}
	\item Примитивен, не предназначен для профессионального рисования.
	\item Не поддерживает Mac OS и Linux.
\end{enumerate}

\subsection{MyPaint}
MyPaint — растровый графический редактор, программа для цифровых ху­дожников. Представляет собой минимальную функциональность графического ре­дактора, неограниченный холст и минимизированный интерфейс на GTK+.\\
MyPaint - MyPaint Contributors,2016,\\
http://mypaint.org/\\


\subsubsection{Достоинства}
\begin{enumerate}
	\item Open source.
	\item Большой набор кистей и их параметров, импорт/экспорт, создание и изменение кистей и их групп. Имеются такие функции кистей как размытие, смешение цветов, стирание и т. д.
	\item Поддержка графического планшета.
	\item Программа доступна для Linux, Windows и Mac OS.
\end{enumerate}

\subsubsection{Недостатки}
\begin{enumerate}
	\item Обладает минимальным интерфейсом и функциями исключительно для рисования, отсутствуют характерные для графических редакторов функции как выделение, масштабирование или фильтры.
\end{enumerate}

%%%Для реализации выберается алгоритм с модификацией отступов, однако ему требуется адаптация для работы с большими кистями. 
% Обратите внимание, что включается не ../dia/..., а inc/dia/...
% В Makefile есть соответствующее правило для inc/dia/*.pdf, которое
% берет исходные файлы из ../dia в этом случае.

%%% Local Variables:
%%% mode: latex
%%% TeX-master: "rpz"
%%% End:

\chapter{Конструкторский раздел}
\label{cha:impl}

\section{Алгоритм отрисовки сплайнами }
Как уже было сказано в аналитической части, оптимальным вариантом сплайна является кубический сплайн, однако у него есть несколько модификаций, для начала рассмотрим общий вид кусочно кубической интерполяции.\cite{bg98} Для сплайна нам нужно чтобы были заданы точки $a = x_1<x_2<...<x_n=b$ и соответствующие им значения $f(x_1),f(x_2)...,f(x_n)$, по ним строится интерполирующая функция $Pf$ таким образом что на каждом отрезке $[x_i,x_i+1], y=1,...,n-1$ она является многочленом $P_i$ степени 3, таким, что 
\begin{center}
	$\begin{array}{c}
	P_i(x_i)=f(x_i),\quad P_i(x_i+1)=f(x_i+1)\\
	P_i`(x_i)=d_i, \quad_i`(x_i+1)=d_i+1 \\
	\end{array}\Bigg\} i= 1,...,n-1$\\
\end{center}
где $d_i, i=1,...,n$ - свободные параметры, тот или иной способ выбора которых определяет метод кусочной интерполяции кубическими многочленами. Полученная функция $Pf$ совпадает с $f$ в точках $x_i, i=1,..,n$ и для любого набора параметров $d_i Pf \in C^{(1)}([a,b])$.\\
Коэффициенты многочлена $P_i$, записанного в форме
\begin{center}
	$P_i(x)=a_{1,i} + a_{2,i}(x-x_i)+a_{3,i}(x-x_i)^2+a_{4,i}(x-x_i)^2(x-x_{i+1})$
\end{center} 
могут быть вычислены по интерполяционной формуле Ньютона с кратными узлами:
\begin{figure}
	[ht]
	\centering
	\includegraphics[width={0,8\textwidth}]{inc/png/nuton}
\end{figure} 
\\Отсюда получаем
\begin{center}
	$\begin{array}{l}
	a_{1,i}=f(x_i)\\
	a_{2,i}=d_i\\
	a_{3,i}=\frac{f(x_i;x_{i+1})-d_i}{x_{i+1} - x_i}\\
	a_{4,i}=\frac{d_i +d_{i+1} - 2f(x_i;x_{i+1})}{(x_{i+1}-x_i)^2}\\
	\end{array}$
\end{center}
, где $f(x_i;x_j) = \frac{f(x_j)-f(x_i)}{x_j-x_i}$ - разделённая разница.\\
Перейдём к другому виду
\begin{center}
	$P_i(x)=c_{1,i} + c_{2,i}(x-x_i)+c_{3,i}(x-x_i)^2+c_{4,i}(x-x_i)^3$
\end{center} 
, где 
\begin{center}
	$\begin{array}{l}
	c_{1,i}=a_{1,i}=f(x_i)\\
	c_{2,i}=a_{2,i}=d_i\\
	c_{3,i}=a_{3,i} - a_{4,i}(x_{i+1}-_i)=\frac{3f(x_i;x_{i+1})-2d_i-d_{i+1}}{x_{i+1} - x_i}\\
	c_{4,i}=a_{4,i}=\frac{d_i +d_{i+1} - 2f(x_i;x_{i+1})}{(x_{i+1}-x_i)^2}\\
	\end{array}$
\end{center}
\subsection{Метод Акимы}
Этот метод приближения используется для борьбы с выбросами приближающей функции, которые появляются, если значения функции в точках заданы с некоторой погрешностью. Поскольку выбросы нежелательны, а разница по сложности вычислений с простым кубическим сплайном незначительна, мы будем использовать этот метод.\cite{aspline70}\cite{comp07}\\
\par Разделённая разность $f(x_{i-1,x_i})$ является приближением к $f`(x_i)$ слева, а $f(x_i,x_{i+1})$ является приближением к $f`(x_i)$ справа. В методе Акимы эти приближения усредняются с весам, котрые тем больше, чем меньше гладкость функции на соседнем отрезке. Окончательная формула для определения параметра $d_i$ имеет вид\\

\begin{center}
	$d_i=
	\begin{cases}
	\frac{w_{i+1}f(x_{i-1};x_i)+w_{i-1}f(x_i;x_{i+1})}{w_{i+1}+w_{i-1}}, \mbox{ если } w_{i+1}^2+w_{i-1}^2 \neq 0\\
	\frac{(x_{i+1}-x_i)f(x_{i-1};x_i)+(x_i-x_{i-1})f(x_i;x_{i+1})}{x_{i+1}-x_{i-1}}, \mbox{ если }  w_{i+1}=w_{i-1}=0
	\end{cases}$
\end{center}
, где $i=3,4,...,n-2$ и $w_j=|f(x_j;x_{j+1})-f(x_{j-1};x_j)|$\\
Для получения недостающих значений $d_1,d_2,d_{n-1},d_n$ точки экстраполируются по формулам.
\begin{center}
	\qquad
	$\begin{array}{l}
	x_0=2x_2-x_4\\
	y_0=(x_1-x_0)(f(x_3;x_2)-2f(x_2;x_1))+y_1\\
	x_1=x_2+x_3-x_4\\
	y_1=(x_4-x_3)(f(x_4;x_3)-2f(x_3;x_2))+y_2\\
	x_{n-2}=x_{n-3}+x_{n-4}-x_{n-5}\\
	y_{n-2}=(x_{n-2}-x_{n-3})(2f(x_{n-3};x_{n-4})-f(x_{n-4};x_{n-3}))+y_{n-3}\\
	x_{n-2}=2x_{n-3}-x_{n-5}\\
	y_{n-2}=(x_{n-1}-x_{n-2})(2f(x_{n-2};x_{n-3})-f(x_{n-3};x_{n-2}))+y_{n-2}\\
	\end{array}$
\end{center}
\subsection{Проблема использования сплайнов при рисовании}
Основное применение сплайнов в компьютерной графике - построение графиков и с этим они хорошо справляются, однако при использовании сплайнов в рисовании у нас могут не выполнятся условия упорядоченности точек по оси Ох и как следствие сплайн не может быть построен. Кроме того, в условиях сильно различающихся изменений по осям Ох и Оу сплайн может рисовать кривую с слишком большим изгибом. Следовательно требуется модифицировать алгоритм рисования сплайна.
\subsection{Предлагаемая модификация }
Модификация должна решать две проблемы:\\
\par Первая - не по каждому набору точке можно построить сплайн, необходимо чтобы по одной из координат они не совпадали и были упорядочены по возрастанию, если представить точки в таком виде возможно, то строится сплайн, иначе точки соединяются прямыми линиями.
\begin{figure}
	\centering
	\includegraphics[width={0,25\textwidth}]{inc/png/badspline}
	\caption{Пример точек, по которым невозможно построить сплайн}
	\label{fig:fig07}
\end{figure} 
\\ \par Вторая - в результате построения сплайна мы можем получить кривую, которая будет иметь сильный изгиб, так называемый выброс, такой результат отрисовывать не стоит, поэтому проверяется выходит ли кривая за границы наименьшего прямоугольника, описывающего входные три точки и рисуется только если не выходит.\\\
\begin{figure}
	\centering
	\includegraphics[width={0,4\textwidth}]{inc/png/badspline2}
	\caption{Выброс}
	\label{fig:fig08}
\end{figure} 
\begin{figure}
	\centering
	\includegraphics[width={0,8\textwidth}]{inc/png/splinediag}
	\caption{Блоксхема модификации}
	\label{fig:fig09}
\end{figure} 


\section{Базовый алгоритм}
В случае задания мазка, проведения им линии до следующей точки при использовании прозрачной кисти мы получим артефакты в виде вдвое меньшей прозрачности в точках из-за наложения линий одной на другую, поэтому алгоритм требуется модифицировать.
\begin{figure}
	\centering
	\includegraphics[width=\textwidth]{inc/png/bas1}
	\caption{Артефакты наложения линий}
	\label{fig:fig10}
\end{figure} 
\subsection{Предлагаемая модификация}
Можно затирать конец прошлой линии, перед рисованием следующей, однако, в таком случае артефакты появляются при пересечении линий.
\begin{lstlisting}[style=pseudocode,caption={Модификация базового алгоритма}]
if (Transparency not full):
	drawPoint(Previous Point, Background Colour)
drawline(Previous Point, Current Point)
\end{lstlisting}

\begin{figure}
	[ht]
	\centering
	\includegraphics[width=\textwidth]{inc/png/basmod}
	\caption{Артефакты модификации}
	\label{fig:fig11}
\end{figure} 
\section{Алгоритм отрисовки мазками}
Для отрисовки непрозрачной кистью будем использовать алгоритм с модификацией отступа 25\%, а для прозрачной кисти наивный алгоритм для баланса между красотой картинки и скоростью работы.

\begin{figure}
	
	\centering
	\includegraphics[width={0,8\textwidth}]{inc/png/moddiag}
	\caption{Алгоритм рисования мазками}
	\label{fig:fig12}
\end{figure} 




\chapter{Технологический раздел}
\label{cha:design}

\section{Выбор парадигмы и языка программирования }

Очевидным выбором для реализации программы стал объектно-ориентированный подход. Главным преимуществом этого подхода, в отличие от структурного, стала возможность не вносить изменений в законченный, рабочий код, а также по мере необходимости делать надстройки над готовой программой. Еще одним плюсом технологии является полиморфизм – возможность использовать любой объект в рамках эксплуатации программного продукта.
Языком, подходящим под поставленную задачу, был выбран С++. Этот язык имеет большое количество возможностей, которые позволяют разрабатывать надежные программные продукты. \\
Язык С++ обладает следующими достоинствами:
\begin{enumerate}
	\item	поддержка объектно-ориентированного подхода, что обеспечивает полиморфизм, наследование и инкапсуляцию при реализации задачи
	\item	использование шаблонов для создания обобщенных контейнеров для разных типов данных
	\item	автоматический вызов деструкторов при уничтожении объектов в порядке, обратном их созданию
	\item	совместимость с языком С, курс по которому был пройден ранее (использовании библиотек на С, поддержка кода на С)
	\item	контроль над структурами и порядком выполнения программы
	\item	наличие перегрузки операторов
\end{enumerate}
Кроме того, язык С++ является самым быстрым среди объектно-ориентированных языков и позволяет при необходимости выполнять достаточно низкоуровневые операции, а также напрямую работать с памятью.
\section{Выбор среды разработки }

В качестве среды разработки был выбран Qt Creator, так как на данный момент он является одной из новейших платформ, отвечающих требованиям проектирования и создания современного ПО, полностью совместим с языком С++, а также библиотеки Qt кроссплатформенные, что позволяет разрабатывать проект как под Microsoft Windows, так и под Linux.


\section{Особенности работы с графическим планшетом}
Когда с планшетом происходит какое-то действие - нажатие на него стилусом, нажатие кнопки на нём или что-то ещё, то он вызывает обрабочик событий и передаёт ему все необходимые данные. В библиотеке Qt обработчику передаётся объект класса QTabletEvent, содержащий всю необходимую информацию о действии. Особенно важными являются методы: pos() - сообщает позицию пера в экранных координатах, pressure - сообщающий уровень давления, type() - позволяющий определить это начало, процесс или конец движения пера по планшету и pointerType() - различающий грифель и ластик.
 

\chapter{Исследовательский раздел}
\label{cha:research}

\section{Условия эксперимента}
Как уже было сказано в аналитической части, мы будем ориентироваться на на работу с графическим планшетом с небольшой частотой опроса, а конкретно с Wacom Bamboo Fun CTE-650, он передаёт до 133 точек в секунду. Поскольку однозначно лучший алгоритм не был выбран, были реализованы три алгоритма: сплайн методом Акимы, алгоритм отрисовки мазками и базовый. Сделаем несколько мазков с отрисовкой базовым алгоритмом, отметками на месте полученных точек и с разной скоростью чтобы проверить наличие проблемы.
\begin{figure}
		\includegraphics[width={\textwidth}]{inc/png/test1}
		\caption{Быстро нарисованная петля}
		\label{fig:test1}
\end{figure}

\begin{figure}
		\includegraphics[width={\textwidth}]{inc/png/test2}
		\caption{Быстрый штрих}
		\label{fig:test2}
\end{figure}

Как видно на рисунках 4.1 и 4.2 просто соединяя точки прямыми при быстром ведении пера мы получим ломаную, а не гладкую кривую.%Добавить воды
\section{Тестовые данные}
Важным является вопрос эталонного результата,ведь провести одну и ту же кривую, но с разными скоростями практически невозможно, ведь человек не может идеально контролировать силу нажатия и траекторию движения пера. Однако мы можем взять математическую функцию и различными способами выбрать точки, которые в ней стоит соединять. Так как скорость проведения пером по планшету может меняться возьмём два варианта выбора точек: с фиксированным шагом аргумента, симулирующую проведение кривой без ускорения и с увеличивающимся в геометрической прогрессии с знаменателем прогрессии 1.2 аргументом, для симуляции ускорения, более быстро возрастающие прогрессии дают бессмысленно плохой результат. Наиболее реалистичным вариантом, показывающим проблему, является диапазон от 25 до 100 точек.
\subsection{Эталонная версия}
Эталонной версией будем считать версию с увеличенным на порядок количеством точек(1000), отрисованных наивным алгоритмом. С таким количеством точек расстояние между несколькими соседними получается не более нескольких пикселей и потому то, что кривая на самом деле состоит из отрезков не заметно.
\subsection{Тестовые функции, эталонные версии }
\begin{figure}
	\begin{minipage}{0.45\textwidth}
		\includegraphics[width={\textwidth}]{inc/png/idfuncs/archspiral}
		\caption{$x(t)=a*a*cos(t)$\\$y(t)=a*t*sin(t)$}
		\label{fig:func3}
	\end{minipage}
	\begin{minipage}{0,45\textwidth}
		\includegraphics[width={\textwidth}]{inc/png/idfuncs/cardioide}
		\caption{$x(t)=a(1-cos(t))cos(t)$\\$y(t)=a*sin(t)(1-cos(t))$}
		\label{fig:func4}
	\end{minipage}
\end{figure}

\begin{figure}
	\begin{minipage}{0.45\textwidth}
		\includegraphics[width={\textwidth}]{inc/png/idfuncs/deltoid}
		\caption{$x(t)=a(2cos(t)+cos(2t)$\\$y(t)=a(2sin(t)-sin(2t)$}
		\label{fig:func5}
	\end{minipage}
	\begin{minipage}{0,45\textwidth}
		\includegraphics[width={\textwidth}]{inc/png/idfuncs/ranunculo}
		\caption{$x(t)=a(6cos(t)-cos(6t))$\\$y(t)=a(6sint(t)-sin(6t)$}
		\label{fig:func6}
	\end{minipage}
\end{figure}

\begin{figure}
	\begin{minipage}{0.45\textwidth}
		\includegraphics[width={\textwidth}]{inc/png/idfuncs/sin}
		\caption{$y=sin(x)$}
		\label{fig:func1}
	\end{minipage}
	\begin{minipage}{0,45\textwidth}
		\includegraphics[width={\textwidth}]{inc/png/idfuncs/sqrt}
		\caption{$y=\sqrt{x}$}
		\label{fig:func2}
	\end{minipage}
\end{figure}


\begin{figure}
\centering
\end{figure}
\begin{figure}
\centering
\includegraphics[width={0,45\textwidth}]{inc/png/idfuncs/trifolium}
\caption{$x(t)=-acos(t)cos(3t)$\\$y(t)=-asin(t)cos(3t)$}
\label{fig:func7}
\end{figure}

\begin{table}
	[ht]
	\resizebox{\textwidth}{0.25\textheight}{
	\begin{tabular}{|p{0.18\textwidth}|p{0.28\textwidth}|p{0.4\textwidth}|}
		\hline
		Эксперимент     & Функция & Расстояние между точками\\
		\hline
		1 & Спираль архимеда & Геометрическая прогрессия\\ \hline
		2 & Спираль архимеда & Фиксированное\\ \hline
		3 & Кардиоида & Геометрическая прогрессия\\ \hline
		4 & Кардиоида & Фиксированное\\ \hline
		5 & Дельтоид & Геометрическая прогрессия\\ \hline
		6 & Дельтоид & Фиксированное\\ \hline
		7 & Ранункулоид & Геометрическая прогрессия\\ \hline
		8 & Ранункулоид & Фиксированное\\ \hline
		9 & Синус & Геометрическая прогрессия\\ \hline
		10 & Синус & Фиксированное\\ \hline
		11 & Квадратный корень & Геометрическая прогрессия\\ \hline
		12 & Квадратный корень & Фиксированное\\ \hline
		13 & Трифолиум & Геометрическая прогрессия\\ \hline
		14 & Трифолиум &Фиксированное\\ 
		\hline
	\end{tabular}
	}
	\caption{Описание экспериментов}
	\label{tab:tabular02}
\end{table}

\subsection{Алгоритм сравнения}
Для сравнения по скорости выполнения выполним каждый из вариантов комбинации алгоритм/набор точек 100 раз, для усреднения результата и замерим суммарное время отрисовки в миллисекундах. Для сравнения совпадения с эталоном мы будем попиксельно сравнивать изображения, подсчитывая число пикселей закрашенных цветом, отличающимся от цвета фона, в переменной $pixels$, суммировать разницу цвета в переменной $diff$, тогда процент совпадения с эталоном может быть посчитана по формуле $\frac{diff}{pixels} * 100\%$

\section{Результаты экспериментов}
В диаграммах совпадения чётные номера экспериментов имеют линейный шаг ускорения, а нечётные шаг с ускорением. В случае шага с ускорением, точек меньше: 18 и 22 против 30 и 60 в случае с линейным шагом т.к. нам нужен одинаковый начальный шаг, а прогрессия с увеличивающимся шагом, очевидно, проходит один и тот же период быстрее. Посчитать количество точек в случае геометрической прогрессии можно по формуле $n=1+\log_{q}(\frac{b_n}{b_1})$, где $b_n$ количество точек при линейном шаге, а $b_1$ начальное значение шага. Полный список результатов экспериментов смотри в приложении 1.
\begin{figure}
	\begin{minipage}{0.45\textwidth}
		\includegraphics[width={\textwidth}]{inc/png/res/CardiodeGeomBas}
		\caption{Базовый алгоритм на 18 точках с ускорением}
		\label{fig:res1}
	\end{minipage}
	\begin{minipage}{0,45\textwidth}
		\includegraphics[width={\textwidth}]{inc/png/res/CardiodeGeomSpl}
		\caption{Сплайн на 18 точках с ускорением}
		\label{fig:res2}
	\end{minipage}
	\begin{minipage}{0.45\textwidth}
		\includegraphics[width={\textwidth}]{inc/png/res/ArchSpiralGeomBas}
		\caption{Базовый алгоритм на 18 точках с ускорением}
		\label{fig:res3}
	\end{minipage}
	\begin{minipage}{0,45\textwidth}
		\includegraphics[width={\textwidth}]{inc/png/res/ArchSpiralGeomSpl}
		\caption{Сплайн на 18 точках с ускорением}
		\label{fig:res4}
	\end{minipage}
\end{figure}
\clearpage
\begin{figure}
	\centering
	\includegraphics[width={\textwidth},height={0,4\textheight}]{inc/png/diags/30dts}
	\caption{Диаграмма совпадений для 30 точек}
	\label{fig:diag1}
	\includegraphics[width={\textwidth},height={0,4\textheight}]{inc/png/diags/60dts}
	\caption{Диаграмма совпадений для 60 точек}
	\label{fig:diag2}
\end{figure}

\begin{figure}
	\centering
	\includegraphics[width={0,9\textwidth}]{inc/png/diags/30dtsperf}
	\caption{Диаграмма производительности для 30 точек}
	\label{fig:diag3}
\end{figure}
\begin{figure}
	\centering
	\includegraphics[width={0,9\textwidth}]{inc/png/diags/60dtsperf}
	\caption{Диаграмма производительности для 60 точек}
	\label{fig:diag4}
\end{figure}
\clearpage

\section{Выводы}
Как мы видим, в значительной части случаев все три алгоритма показывают схожие результаты совпадения с эталонной версией, однако достаточно часто сплайн показывает заметно больший процент совпадения с эталонной версией(до 30\% разницы с базовым алгоритмом), но при этом по скорости он на порядок проигрывает базовому, а алгоритму мазками в зависимости от эксперимента проигрывает в производительности от нескольких процентов до 5 раз, однако даже в худшем случае мы получаем менее миллисекунды на построение связи между двумя точками, значит его производительность достаточна для комфортной интерактивной работы. 


\section{Заключение}
В рамках данной работы нами был проведён анализ существующих алгоритмов, используемых при соединении точек, три из них были успешно модифицированы и реализованы, после чего был проведено исследование скорости их работы и совпадения с эталонной версией отрисовки. Также в рамках исследования были подтверждены теоретические преимущества и недостатки алгоритмов. Разработанное ПО успешно выполняет поставленную задачу - отрисовку линий, вводимых с помощью графического планшета.\\
Основными возможными векторами дальнейшего развития данного ПО являются: 
\begin{enumerate}
	\item	Дальнейшее исследование работы со сплайнами
	\item	Добавление других динамических параметров
	\item	Поддержка графических планшетов с датчиком наклона пера
	\item	Поддержка других устройств ввода
	\item	Симуляция реалистичных кистей
\end{enumerate}





\backmatter %% Здесь заканчивается нумерованная часть документа и начинаются ссылки и
            %% заключение

%\Conclusion % заключение к отчёту

В процессе учебной практики я провел анализ алгоритмов отрисовки кистей. Были рассмотрены основные проблемы рисования кистью, выбран алгоритм и язык программирования, наиболее подходящие для решения поставленной задачи.  
 В связи с выбранным объектно-ориентированным разработка проекта будет происходить поэтапно, что позволит минимизировать количество возникающих ошибок и улучшить качество кода. Кроме того я получил опыт в составлении РПЗ, а именно аналитического и конструкторского разделов.


%%% Local Variables: 
%%% mode: latex
%%% TeX-master: "rpz"
%%% End: 


% % Список литературы при помощи BibTeX
% Юзать так:
%
% pdflatex rpz
% bibtex rpz
% pdflatex rpz

\bibliographystyle{gost780u}
\bibliography{rpz}


%%% Local Variables: 
%%% mode: latex
%%% TeX-master: "rpz"
%%% End: 


\appendix   % Тут идут приложения

\chapter{Все результаты отрисовки}
\label{cha:appendix1}

\section{Эксперимент с 30ю точками}
\subsection{Базовый}
\begin{figure}  
	\begin{minipage}{0,5\textwidth}
		\includegraphics[width={\textwidth}]{inc/png/30dts/Basic/ArchSpiralGeom}
		\label{fig:app1}
		\caption{Эксперимент 1}
	\end{minipage}
	\begin{minipage}{0,5\textwidth}
		\includegraphics[width={\textwidth}]{inc/png/30dts/Basic/ArchSpiralLin}
		\label{fig:app2}
		\caption{Эксперимент 2}
	\end{minipage}
	
	\begin{minipage}{0,5\textwidth}
		\includegraphics[width={\textwidth}]{inc/png/30dts/Basic/CardiodeGeom}
		\label{fig:app3}
		\caption{Эксперимент 3}
	\end{minipage}
	\begin{minipage}{0,5\textwidth}
		\includegraphics[width={\textwidth}]{inc/png/30dts/Basic/CardiodeLin}
		\label{fig:app4}
		\caption{Эксперимент 4}
	\end{minipage}
	
	\begin{minipage}{0,5\textwidth}
		\includegraphics[width={\textwidth}]{inc/png/30dts/Basic/DeltoidGeom}
		\label{fig:app1}
		\caption{Эксперимент 5}
	\end{minipage}
	\begin{minipage}{0,5\textwidth}
		\includegraphics[width={\textwidth}]{inc/png/30dts/Basic/DeltoidLin}
		\label{fig:app2}
		\caption{Эксперимент 6}
	\end{minipage}
\end{figure}


\begin{figure}  
	\begin{minipage}{0,5\textwidth}
		\includegraphics[width={\textwidth}]{inc/png/30dts/Basic/RanunculoidGeom}
		\label{fig:app1}
		\caption{Эксперимент 7}
	\end{minipage}
	\begin{minipage}{0,5\textwidth}
		\includegraphics[width={\textwidth}]{inc/png/30dts/Basic/RanunculoidLin}
		\label{fig:app2}
		\caption{Эксперимент 8}
	\end{minipage}
	
	\begin{minipage}{0,5\textwidth}
		\includegraphics[width={\textwidth}]{inc/png/30dts/Basic/SinGeom}
		\label{fig:app1}
		\caption{Эксперимент 9}
	\end{minipage}
	\begin{minipage}{0,5\textwidth}
		\includegraphics[width={\textwidth}]{inc/png/30dts/Basic/SinLin}
		\label{fig:app2}
		\caption{Эксперимент 10}
	\end{minipage}
	
	\begin{minipage}{0,5\textwidth}
		\includegraphics[width={\textwidth}]{inc/png/30dts/Basic/SqrtGeom}
		\label{fig:app1}
		\caption{Эксперимент 11}
	\end{minipage}
	\begin{minipage}{0,5\textwidth}
		\includegraphics[width={\textwidth}]{inc/png/30dts/Basic/SqrtLin}
		\label{fig:app2}
		\caption{Эксперимент 12}
	\end{minipage}
	
	\begin{minipage}{0,5\textwidth}
		\includegraphics[width={\textwidth}]{inc/png/30dts/Basic/TrifoliumGeom}
		\label{fig:app1}
		\caption{Эксперимент 13}
	\end{minipage}
	\begin{minipage}{0,5\textwidth}
		\includegraphics[width={\textwidth}]{inc/png/30dts/Basic/TrifoliumLin}
		\label{fig:app2}
		\caption{Эксперимент 14}
	\end{minipage}
\end{figure}
\clearpage

\subsection{Мазок}
\begin{figure}  
	\begin{minipage}{0,5\textwidth}
		\includegraphics[width={\textwidth}]{inc/png/30dts/Mod/ArchSpiralGeom}
		\label{fig:app1}
		\caption{Эксперимент 1}
	\end{minipage}
	\begin{minipage}{0,5\textwidth}
		\includegraphics[width={\textwidth}]{inc/png/30dts/Mod/ArchSpiralLin}
		\label{fig:app2}
		\caption{Эксперимент 2}
	\end{minipage}
	
	\begin{minipage}{0,5\textwidth}
		\includegraphics[width={\textwidth}]{inc/png/30dts/Mod/CardiodeGeom}
		\label{fig:app3}
		\caption{Эксперимент 3}
	\end{minipage}
	\begin{minipage}{0,5\textwidth}
		\includegraphics[width={\textwidth}]{inc/png/30dts/Mod/CardiodeLin}
		\label{fig:app4}
		\caption{Эксперимент 4}
	\end{minipage}
	
	\begin{minipage}{0,5\textwidth}
		\includegraphics[width={\textwidth}]{inc/png/30dts/Mod/DeltoidGeom}
		\label{fig:app1}
		\caption{Эксперимент 5}
	\end{minipage}
	\begin{minipage}{0,5\textwidth}
		\includegraphics[width={\textwidth}]{inc/png/30dts/Mod/DeltoidLin}
		\label{fig:app2}
		\caption{Эксперимент 6}
	\end{minipage}
\end{figure}


\begin{figure}  
	\begin{minipage}{0,5\textwidth}
		\includegraphics[width={\textwidth}]{inc/png/30dts/Mod/RanunculoidGeom}
		\label{fig:app1}
		\caption{Эксперимент 7}
	\end{minipage}
	\begin{minipage}{0,5\textwidth}
		\includegraphics[width={\textwidth}]{inc/png/30dts/Mod/RanunculoidLin}
		\label{fig:app2}
		\caption{Эксперимент 8}
	\end{minipage}
	
	\begin{minipage}{0,5\textwidth}
		\includegraphics[width={\textwidth}]{inc/png/30dts/Mod/SinGeom}
		\label{fig:app1}
		\caption{Эксперимент 9}
	\end{minipage}
	\begin{minipage}{0,5\textwidth}
		\includegraphics[width={\textwidth}]{inc/png/30dts/Mod/SinLin}
		\label{fig:app2}
		\caption{Эксперимент 10}
	\end{minipage}
	
	\begin{minipage}{0,5\textwidth}
		\includegraphics[width={\textwidth}]{inc/png/30dts/Mod/SqrtGeom}
		\label{fig:app1}
		\caption{Эксперимент 11}
	\end{minipage}
	\begin{minipage}{0,5\textwidth}
		\includegraphics[width={\textwidth}]{inc/png/30dts/Mod/SqrtLin}
		\label{fig:app2}
		\caption{Эксперимент 12}
	\end{minipage}
	
	\begin{minipage}{0,5\textwidth}
		\includegraphics[width={\textwidth}]{inc/png/30dts/Mod/TrifoliumGeom}
		\label{fig:app1}
		\caption{Эксперимент 13}
	\end{minipage}
	\begin{minipage}{0,5\textwidth}
		\includegraphics[width={\textwidth}]{inc/png/30dts/Mod/TrifoliumLin}
		\label{fig:app2}
		\caption{Эксперимент 14}
	\end{minipage}
\end{figure}
\clearpage


\subsection{Сплайн}
\begin{figure}  
	\begin{minipage}{0,5\textwidth}
		\includegraphics[width={\textwidth}]{inc/png/30dts/Spline/ArchSpiralGeom}
		\label{fig:app1}
		\caption{Эксперимент 1}
	\end{minipage}
	\begin{minipage}{0,5\textwidth}
		\includegraphics[width={\textwidth}]{inc/png/30dts/Spline/ArchSpiralLin}
		\label{fig:app2}
		\caption{Эксперимент 2}
	\end{minipage}
	
	\begin{minipage}{0,5\textwidth}
		\includegraphics[width={\textwidth}]{inc/png/30dts/Spline/CardiodeGeom}
		\label{fig:app3}
		\caption{Эксперимент 3}
	\end{minipage}
	\begin{minipage}{0,5\textwidth}
		\includegraphics[width={\textwidth}]{inc/png/30dts/Spline/CardiodeLin}
		\label{fig:app4}
		\caption{Эксперимент 4}
	\end{minipage}
	
	\begin{minipage}{0,5\textwidth}
		\includegraphics[width={\textwidth}]{inc/png/30dts/Spline/DeltoidGeom}
		\label{fig:app1}
		\caption{Эксперимент 5}
	\end{minipage}
	\begin{minipage}{0,5\textwidth}
		\includegraphics[width={\textwidth}]{inc/png/30dts/Spline/DeltoidLin}
		\label{fig:app2}
		\caption{Эксперимент 6}
	\end{minipage}
\end{figure}


\begin{figure}  
	\begin{minipage}{0,5\textwidth}
		\includegraphics[width={\textwidth}]{inc/png/30dts/Spline/RanunculoidGeom}
		\label{fig:app1}
		\caption{Эксперимент 7}
	\end{minipage}
	\begin{minipage}{0,5\textwidth}
		\includegraphics[width={\textwidth}]{inc/png/30dts/Spline/RanunculoidLin}
		\label{fig:app2}
		\caption{Эксперимент 8}
	\end{minipage}
	
	\begin{minipage}{0,5\textwidth}
		\includegraphics[width={\textwidth}]{inc/png/30dts/Spline/SinGeom}
		\label{fig:app1}
		\caption{Эксперимент 9}
	\end{minipage}
	\begin{minipage}{0,5\textwidth}
		\includegraphics[width={\textwidth}]{inc/png/30dts/Spline/SinLin}
		\label{fig:app2}
		\caption{Эксперимент 10}
	\end{minipage}
	
	\begin{minipage}{0,5\textwidth}
		\includegraphics[width={\textwidth}]{inc/png/30dts/Spline/SqrtGeom}
		\label{fig:app1}
		\caption{Эксперимент 11}
	\end{minipage}
	\begin{minipage}{0,5\textwidth}
		\includegraphics[width={\textwidth}]{inc/png/30dts/Spline/SqrtLin}
		\label{fig:app2}
		\caption{Эксперимент 12}
	\end{minipage}
	
	\begin{minipage}{0,5\textwidth}
		\includegraphics[width={\textwidth}]{inc/png/30dts/Spline/TrifoliumGeom}
		\label{fig:app1}
		\caption{Эксперимент 13}
	\end{minipage}
	\begin{minipage}{0,5\textwidth}
		\includegraphics[width={\textwidth}]{inc/png/30dts/Spline/TrifoliumLin}
		\label{fig:app2}
		\caption{Эксперимент 14}
	\end{minipage}
\end{figure}
\clearpage


\section{Эксперимент с 60ю точками}
\subsection{Базовый}
\begin{figure}  
	\begin{minipage}{0,5\textwidth}
		\includegraphics[width={\textwidth}]{inc/png/60dts/Basic/ArchSpiralGeom}
		\label{fig:app1}
		\caption{Эксперимент 1}
	\end{minipage}
	\begin{minipage}{0,5\textwidth}
		\includegraphics[width={\textwidth}]{inc/png/60dts/Basic/ArchSpiralLin}
		\label{fig:app2}
		\caption{Эксперимент 2}
	\end{minipage}
	
	\begin{minipage}{0,5\textwidth}
		\includegraphics[width={\textwidth}]{inc/png/60dts/Basic/CardiodeGeom}
		\label{fig:app3}
		\caption{Эксперимент 3}
	\end{minipage}
	\begin{minipage}{0,5\textwidth}
		\includegraphics[width={\textwidth}]{inc/png/60dts/Basic/CardiodeLin}
		\label{fig:app4}
		\caption{Эксперимент 4}
	\end{minipage}
	
	\begin{minipage}{0,5\textwidth}
		\includegraphics[width={\textwidth}]{inc/png/60dts/Basic/DeltoidGeom}
		\label{fig:app1}
		\caption{Эксперимент 5}
	\end{minipage}
	\begin{minipage}{0,5\textwidth}
		\includegraphics[width={\textwidth}]{inc/png/60dts/Basic/DeltoidLin}
		\label{fig:app2}
		\caption{Эксперимент 6}
	\end{minipage}
\end{figure}


\begin{figure}  
	\begin{minipage}{0,5\textwidth}
		\includegraphics[width={\textwidth}]{inc/png/60dts/Basic/RanunculoidGeom}
		\label{fig:app1}
		\caption{Эксперимент 7}
	\end{minipage}
	\begin{minipage}{0,5\textwidth}
		\includegraphics[width={\textwidth}]{inc/png/60dts/Basic/RanunculoidLin}
		\label{fig:app2}
		\caption{Эксперимент 8}
	\end{minipage}
	
	\begin{minipage}{0,5\textwidth}
		\includegraphics[width={\textwidth}]{inc/png/60dts/Basic/SinGeom}
		\label{fig:app1}
		\caption{Эксперимент 9}
	\end{minipage}
	\begin{minipage}{0,5\textwidth}
		\includegraphics[width={\textwidth}]{inc/png/60dts/Basic/SinLin}
		\label{fig:app2}
		\caption{Эксперимент 10}
	\end{minipage}
	
	\begin{minipage}{0,5\textwidth}
		\includegraphics[width={\textwidth}]{inc/png/60dts/Basic/SqrtGeom}
		\label{fig:app1}
		\caption{Эксперимент 11}
	\end{minipage}
	\begin{minipage}{0,5\textwidth}
		\includegraphics[width={\textwidth}]{inc/png/60dts/Basic/SqrtLin}
		\label{fig:app2}
		\caption{Эксперимент 12}
	\end{minipage}
	
	\begin{minipage}{0,5\textwidth}
		\includegraphics[width={\textwidth}]{inc/png/60dts/Basic/TrifoliumGeom}
		\label{fig:app1}
		\caption{Эксперимент 13}
	\end{minipage}
	\begin{minipage}{0,5\textwidth}
		\includegraphics[width={\textwidth}]{inc/png/60dts/Basic/TrifoliumLin}
		\label{fig:app2}
		\caption{Эксперимент 14}
	\end{minipage}
\end{figure}
\clearpage

\subsection{Мазок}
\begin{figure}  
	\begin{minipage}{0,5\textwidth}
		\includegraphics[width={\textwidth}]{inc/png/60dts/Mod/ArchSpiralGeom}
		\label{fig:app1}
		\caption{Эксперимент 1}
	\end{minipage}
	\begin{minipage}{0,5\textwidth}
		\includegraphics[width={\textwidth}]{inc/png/60dts/Mod/ArchSpiralLin}
		\label{fig:app2}
		\caption{Эксперимент 2}
	\end{minipage}
	
	\begin{minipage}{0,5\textwidth}
		\includegraphics[width={\textwidth}]{inc/png/60dts/Mod/CardiodeGeom}
		\label{fig:app3}
		\caption{Эксперимент 3}
	\end{minipage}
	\begin{minipage}{0,5\textwidth}
		\includegraphics[width={\textwidth}]{inc/png/60dts/Mod/CardiodeLin}
		\label{fig:app4}
		\caption{Эксперимент 4}
	\end{minipage}
	
	\begin{minipage}{0,5\textwidth}
		\includegraphics[width={\textwidth}]{inc/png/60dts/Mod/DeltoidGeom}
		\label{fig:app1}
		\caption{Эксперимент 5}
	\end{minipage}
	\begin{minipage}{0,5\textwidth}
		\includegraphics[width={\textwidth}]{inc/png/60dts/Mod/DeltoidLin}
		\label{fig:app2}
		\caption{Эксперимент 6}
	\end{minipage}
\end{figure}


\begin{figure}  
	\begin{minipage}{0,5\textwidth}
		\includegraphics[width={\textwidth}]{inc/png/60dts/Mod/RanunculoidGeom}
		\label{fig:app1}
		\caption{Эксперимент 7}
	\end{minipage}
	\begin{minipage}{0,5\textwidth}
		\includegraphics[width={\textwidth}]{inc/png/60dts/Mod/RanunculoidLin}
		\label{fig:app2}
		\caption{Эксперимент 8}
	\end{minipage}
	
	\begin{minipage}{0,5\textwidth}
		\includegraphics[width={\textwidth}]{inc/png/60dts/Mod/SinGeom}
		\label{fig:app1}
		\caption{Эксперимент 9}
	\end{minipage}
	\begin{minipage}{0,5\textwidth}
		\includegraphics[width={\textwidth}]{inc/png/60dts/Mod/SinLin}
		\label{fig:app2}
		\caption{Эксперимент 10}
	\end{minipage}
	
	\begin{minipage}{0,5\textwidth}
		\includegraphics[width={\textwidth}]{inc/png/60dts/Mod/SqrtGeom}
		\label{fig:app1}
		\caption{Эксперимент 11}
	\end{minipage}
	\begin{minipage}{0,5\textwidth}
		\includegraphics[width={\textwidth}]{inc/png/60dts/Mod/SqrtLin}
		\label{fig:app2}
		\caption{Эксперимент 12}
	\end{minipage}
	
	\begin{minipage}{0,5\textwidth}
		\includegraphics[width={\textwidth}]{inc/png/60dts/Mod/TrifoliumGeom}
		\label{fig:app1}
		\caption{Эксперимент 13}
	\end{minipage}
	\begin{minipage}{0,5\textwidth}
		\includegraphics[width={\textwidth}]{inc/png/60dts/Mod/TrifoliumLin}
		\label{fig:app2}
		\caption{Эксперимент 14}
	\end{minipage}
\end{figure}
\clearpage


\subsection{Сплайн}
\begin{figure}  
	\begin{minipage}{0,5\textwidth}
		\includegraphics[width={\textwidth}]{inc/png/60dts/Spline/ArchSpiralGeom}
		\label{fig:app1}
		\caption{Эксперимент 1}
	\end{minipage}
	\begin{minipage}{0,5\textwidth}
		\includegraphics[width={\textwidth}]{inc/png/60dts/Spline/ArchSpiralLin}
		\label{fig:app2}
		\caption{Эксперимент 2}
	\end{minipage}
	
	\begin{minipage}{0,5\textwidth}
		\includegraphics[width={\textwidth}]{inc/png/60dts/Spline/CardiodeGeom}
		\label{fig:app3}
		\caption{Эксперимент 3}
	\end{minipage}
	\begin{minipage}{0,5\textwidth}
		\includegraphics[width={\textwidth}]{inc/png/60dts/Spline/CardiodeLin}
		\label{fig:app4}
		\caption{Эксперимент 4}
	\end{minipage}
	
	\begin{minipage}{0,5\textwidth}
		\includegraphics[width={\textwidth}]{inc/png/60dts/Spline/DeltoidGeom}
		\label{fig:app1}
		\caption{Эксперимент 5}
	\end{minipage}
	\begin{minipage}{0,5\textwidth}
		\includegraphics[width={\textwidth}]{inc/png/60dts/Spline/DeltoidLin}
		\label{fig:app2}
		\caption{Эксперимент 6}
	\end{minipage}
\end{figure}


\begin{figure}  
	\begin{minipage}{0,5\textwidth}
		\includegraphics[width={\textwidth}]{inc/png/60dts/Spline/RanunculoidGeom}
		\label{fig:app1}
		\caption{Эксперимент 7}
	\end{minipage}
	\begin{minipage}{0,5\textwidth}
		\includegraphics[width={\textwidth}]{inc/png/60dts/Spline/RanunculoidLin}
		\label{fig:app2}
		\caption{Эксперимент 8}
	\end{minipage}
	
	\begin{minipage}{0,5\textwidth}
		\includegraphics[width={\textwidth}]{inc/png/60dts/Spline/SinGeom}
		\label{fig:app1}
		\caption{Эксперимент 9}
	\end{minipage}
	\begin{minipage}{0,5\textwidth}
		\includegraphics[width={\textwidth}]{inc/png/60dts/Spline/SinLin}
		\label{fig:app2}
		\caption{Эксперимент 10}
	\end{minipage}
	
	\begin{minipage}{0,5\textwidth}
		\includegraphics[width={\textwidth}]{inc/png/60dts/Spline/SqrtGeom}
		\label{fig:app1}
		\caption{Эксперимент 11}
	\end{minipage}
	\begin{minipage}{0,5\textwidth}
		\includegraphics[width={\textwidth}]{inc/png/60dts/Spline/SqrtLin}
		\label{fig:app2}
		\caption{Эксперимент 12}
	\end{minipage}
	
	\begin{minipage}{0,5\textwidth}
		\includegraphics[width={\textwidth}]{inc/png/60dts/Spline/TrifoliumGeom}
		\label{fig:app1}
		\caption{Эксперимент 13}
	\end{minipage}
	\begin{minipage}{0,5\textwidth}
		\includegraphics[width={\textwidth}]{inc/png/60dts/Spline/TrifoliumLin}
		\label{fig:app2}
		\caption{Эксперимент 14}
	\end{minipage}
\end{figure}
\clearpage

%\chapter{Еще картинки}
\label{cha:appendix2}

\begin{figure}
\centering
\caption{Еще одна картинка, ничем не лучше предыдущей. Но надо же как-то заполнить место.}
\end{figure}

%%% Local Variables: 
%%% mode: latex
%%% TeX-master: "rpz"
%%% End: 


\end{document}

%%% Local Variables:
%%% mode: latex
%%% TeX-master: t
%%% End:
